\chapter*{Preface}
\chaptermark{Preface}

Dr. Vinod Reddy used to teach us a course on \textit{Design and Analysis of Algorithms} during the first semester of our \textit{Master of Technology} (MTech) coursework at the Indian Institute of technology Vadodara (IIITV). He introduced me to the graph burning problems and several other graph problems and games which are NP-Hard for general graphs.

He, in one of his assignments, asked us to write an algorithm on the optimal \textit{burning} of simple paths and \textit{interval graphs}. Although, how to compute the \textit{burning number} of paths had already been published but there was no data for \textit{burning} of \textit{interval graphs}. I proposed him a rough solution of how to \textit{burn} \textit{interval graphs}. After the semester, Dr. Reddy left IIITV; he joined for an assistant professorship position at the Indian Institute of Technology Bhilai.

Initially, I wanted to work in computational biology but that did not work out. I eventually started working on graph algorithms with Dr. Swapnil Lokhande and Dr. Kaushik Mondal at IIITV. The more we got interested, the more we studied and discussed on the problems and prospective solutions.

We improvised on the results that I presented to Dr. Reddy. We also worked out some more interesting results. This document is an extract of our study and findings. We have included a good amount of research that has been done on \textit{graph burning} so far. This document is a dissertation for the partial fulfilment of my MTech degree.

We discuss briefly an overview of the (first) two introductory chapters as follows.

An overview of the remaining chapters is present in \Cref{section:organization-of-chapters} in \Cref{chapter:introduction} after we introduce some basic concepts which shall be helpful in following with the rest of the chapters.

The following chapters include work which has already been published, and also some of our original findings related to graph burning. At all places where we discuss on some published works or derive excerpts to our text, we have given citations so that the reader can refer and study those works in detail.

I hope that the reader finds the collection of the following chapters interesting and easy to comprehend. Comments are always welcome.\\~\\

\begin{flushright}
\textsc{Arya Tanmay Gupta}
\end{flushright}